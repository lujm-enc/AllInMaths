% style.tex
% Định dạng hiện đại, đẹp mắt

% ---- 1. Color Palette ----
% A professional, modern color scheme. Easy to change here.
\definecolor{myblue}{HTML}{005EA2}      % A deep, professional blue
\definecolor{mygreen}{HTML}{008060}     % A calm, academic green
\definecolor{myorange}{HTML}{F47738}    % A soft, warning orange
\definecolor{mypurple}{HTML}{6F77B2}    % A gentle purple
\definecolor{myred}{HTML}{D4351C}       % A clear red for warnings/notes
\definecolor{mygray}{HTML}{F3F2F1}      % A very light gray for box backgrounds
\definecolor{darkgray}{HTML}{505A5F}    % For text

% ---- 2. Typography ----
% Ensure you have installed "Source Serif 4" and "Inter" fonts
\setmainfont{Source Serif 4}[      % <<< THIS LINE IS UPDATED
    UprightFont = *,
    BoldFont = * SemiBold,
    ItalicFont = * Italic,
    BoldItalicFont = * SemiBold Italic
]
\setsansfont{Inter}[
    UprightFont = * Regular,
    BoldFont = * Bold,
    ItalicFont = * Italic,
    BoldItalicFont = * Bold Italic
]
% \setmonofont{Fira Code} % Optional: A great font for code listings

% Set default text color
\color{darkgray}

% ---- 3. Page Layout (Headers & Footers) ----
\pagestyle{fancy}
\fancyhf{} % Clear all header and footer fields
\fancyhead[LE,RO]{\sffamily\bfseries\thepage}
\fancyhead[LO]{\sffamily\nouppercase{\rightmark}}
\fancyhead[RE]{\sffamily\nouppercase{\leftmark}}
\renewcommand{\headrulewidth}{0.5pt}
\renewcommand{\footrulewidth}{0pt}
\fancypagestyle{plain}{ % For chapter start pages
    \fancyhf{}
    \renewcommand{\headrulewidth}{0pt}
}

% ---- 4. Chapter and Section Titles (Modernized & Corrected) ----
% This version uses the correct titlesec syntax for BOTH chapters and sections.

% --- MODERN CHAPTER STYLE ---
% This command is already correct and robust.
\titleformat{\chapter}[display]
  {\sffamily\bfseries\color{darkgray}} % Global format for the chapter title block
  {\filleft\huge\color{myblue}Chương \thechapter} % The label part (e.g., "Chương 1")
  {10pt} % Vertical separation between the label and the title
  { % Code that comes BEFORE the title text
    \color{myblue}\titlerule[2pt]
    \vspace{8pt}
    \Huge\raggedleft % Apply formatting to the title text itself
  }
  [] % Code that comes AFTER the title
\titlespacing*{\chapter}{0pt}{-20pt}{50pt} % Adjusts spacing (left, before, after)


% --- MODERN SECTION STYLE ---
% The illegal #1 has been removed, fixing the compilation error.
\titleformat{\section}
  {\sffamily\Large\bfseries\color{myblue!90!black}} % Font for the entire title line
  {\color{myblue}\vrule width 2pt height 1.1em depth 0.2em\hspace{0.6em}\thesection} % The label
  {0.5em} % Horizontal space
  {}      % <<< THE FIX IS HERE. This was "{#1}" and is now correctly empty.
\titlespacing*{\section}{0pt}{30pt}{15pt} % Adjust vertical spacing (left, before, after)

% ---- 5. Table of Contents ----
\titlecontents{chapter}[1.5em]
  {\sffamily\bfseries\large\vspace{10pt}}
  {Chương \thecontentslabel:\quad}
  {}
  {\sffamily\bfseries\hfill\contentspage}

\titlecontents{section}[3.8em]
  {\sffamily\small}
  {\thecontentslabel.\quad}
  {}
  {\sffamily\hfill\contentspage}

% ---- 6. Modern Environments: A Definitive and Stable Architecture ----

% --- PART 1: The Master "Theory" Box ---
% A large, breakable half-box for theoretical content.
\newtcolorbox{lythuyetbox}[1]{
    breakable,
    enhanced,
    frame empty,
    colback=white,
    fonttitle=\sffamily\bfseries\Large,
    coltitle=myblue, % <<< THE FIX: Explicitly sets the title's text color.
    % The "half-box" border
    borderline north={1.5pt}{0pt}{myblue},
    borderline west={1.5pt}{0pt}{myblue},
    % Title placement and styling
    attach boxed title to top left={
        xshift=1cm,
        yshift=-0.5\baselineskip, % Centers the title on the top line
    },
    boxed title style={
        frame empty,
        colback=white, % Makes the title block opaque to hide the line behind it
    },
    title=#1,
    % Spacing for the content inside
    left=6mm, top=8mm, bottom=4mm, right=4mm,
}


% --- PART 2: DEFINITIVE, STABLE ENVIRONMENTS ---
% This is the final, correct solution. It uses the "titlerule=0pt" key to
% reliably remove the dividing line, achieving the integrated title design.

% --- Environment Definitions ---

\newtcbtheorem[auto counter, number within=chapter]{dinhnghia}{Định nghĩa}
  {
    enhanced, frame empty, sharp corners,
    % Make the title and content backgrounds the same color
    colback=myorange!10!white,
    colbacktitle=myorange!10!white,
    % Style the title text
    coltitle=myorange!35!black, fonttitle=\sffamily\bfseries,
    % CORRECT, STABLE KEY: Sets the thickness of the rule under the title to zero.
    titlerule=0pt,
    % Standard options for layout
    boxsep=1.5mm, left=5mm, right=5mm, top=3mm, bottom=3mm,
    title={\icon{book}\hspace{0.5em}Định nghĩa~\thetcbcounter\kvtcb@ifblank{#2}{}{ : #2}}
  }
  {}

\newtcbtheorem[auto counter, number within=chapter]{dinhly}{Định lý}
  {
    enhanced, frame empty, sharp corners,
    colback=myblue!10!white, colbacktitle=myblue!10!white,
    coltitle=myblue!35!black, fonttitle=\sffamily\bfseries,
    titlerule=0pt,
    boxsep=1.5mm, left=5mm, right=5mm, top=3mm, bottom=3mm,
    title={\icon{landmark}\hspace{0.5em}Định lý~\thetcbcounter\kvtcb@ifblank{#2}{}{ : #2}}
  }
  {}

\newtcbtheorem[auto counter, number within=chapter]{vidu}{Ví dụ}
  {
    enhanced, frame empty, sharp corners,
    colback=mygreen!10!white, colbacktitle=mygreen!10!white,
    coltitle=mygreen!35!black, fonttitle=\sffamily\bfseries,
    titlerule=0pt,
    boxsep=1.5mm, left=5mm, right=5mm, top=3mm, bottom=3mm,
    title={\icon{pencil-alt}\hspace{0.5em}Ví dụ~\thetcbcounter\kvtcb@ifblank{#2}{}{ : #2}}
  }
  {}

\newtcbtheorem[auto counter, number within=chapter]{hequa}{Hệ quả}
  {
    enhanced, frame empty, sharp corners,
    colback=mypurple!10!white, colbacktitle=mypurple!10!white,
    coltitle=mypurple!35!black, fonttitle=\sffamily\bfseries,
    titlerule=0pt,
    boxsep=1.5mm, left=5mm, right=5mm, top=3mm, bottom=3mm,
    title={\icon{star}\hspace{0.5em}Hệ quả~\thetcbcounter\kvtcb@ifblank{#2}{}{ : #2}}
  }
  {}

\newtcbtheorem[auto counter, number within=chapter]{chuy}{Chú ý}
  {
    enhanced, frame empty, sharp corners,
    colback=myred!15!white, colbacktitle=myred!15!white,
    coltitle=myred!40!black, fonttitle=\sffamily\bfseries,
    titlerule=0pt,
    boxsep=1.5mm, left=5mm, right=5mm, top=3mm, bottom=3mm,
    title={\icon{exclamation-triangle}\hspace{0.5em}Chú ý~\thetcbcounter\kvtcb@ifblank{#2}{}{ : #2}}
  }
  {}


% --- PART 3: The Exercise and Solution System (Modern & Simplistic) ---
% This new design focuses on minimalism and readability.

% Define a counter for the exercises, which resets every chapter.
\newcounter{cauhoicounter}[chapter]
\renewcommand{\thecauhoicounter}{\thechapter.\arabic{cauhoicounter}}

% \NewDocumentEnvironment requires the xparse package (already loaded in packages.tex)
% This new 'cauhoi' environment has an optional title and a mandatory label.
% Usage: \begin{cauhoi}[Optional Title]{label-name} ... \end{cauhoi}
\NewDocumentEnvironment{cauhoi}{ o m }{% o = Optional arg [in brackets], m = Mandatory arg {in braces}
    \refstepcounter{cauhoicounter}\label{#2}% Step and set the label for \ref
    \begin{tcolorbox}[
        enhanced, colback=white, colframe=darkgray, boxrule=0.5pt, sharp corners
    ]
    \sffamily\bfseries\color{darkgray}% Styling for the title line
    Bài tập~\thecauhoicounter
    \IfValueT{#1}{% If the optional title #1 was provided, print it with a colon
        : #1%
    }%
    \par\nobreak\vspace{1.5mm}% Line break after the title line
    \normalfont\color{darkgray} % Reset font for the main question text
}{
    \end{tcolorbox}
}

% This new 'loigiai' environment is a simple, borderless block.
\NewDocumentEnvironment{loigiai}{}{% No arguments needed
    \begin{tcolorbox}[
        enhanced, frame empty, % No border
        colback=mygray,      % Light gray background
        sharp corners,
        top=8pt, bottom=8pt, left=8pt, right=8pt % Padding
    ]
    \sffamily\bfseries\color{mygreen!30!black}% Styling for "Lời giải"
    Lời giải
    \par\nobreak\vspace{1.5mm}% Line break
    \normalfont\color{darkgray}% Reset font for the solution text
}{
    \end{tcolorbox}
}