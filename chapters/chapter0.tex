% chapter1.tex
\chapter{Giới thiệu}
\label{ch:intro}

\section{Mục đích}
Sách này nhằm cung cấp kiến thức toán học chuyên sâu.

% --- USING THE FINAL, CORRECT, AND SIMPLE SYNTAX ---
\begin{lythuyetbox}{Các khái niệm cơ bản về Số học}

\begin{dinhnghia}{Số nguyên tố}{dn:so-nguyen-to}
Một số nguyên lớn hơn 1 chỉ có hai ước là 1 và chính nó được gọi là số nguyên tố.
\end{dinhnghia}

Dựa trên Định nghĩa \ref{dn:so-nguyen-to}, ta có định lý quan trọng sau đây.

\begin{dinhly}{Định lý cơ bản của số học}{dl:co-ban-so-hoc}
Mọi số nguyên lớn hơn 1 đều có thể được biểu diễn duy nhất (không kể thứ tự các thừa số) dưới dạng tích của các số nguyên tố.
\end{dinhly}

Từ Định lý \ref{dl:co-ban-so-hoc}, ta có một hệ quả trực tiếp.

\begin{hequa}{}{hq:vo-han-snt}
Có vô số các số nguyên tố.
\end{hequa}

\begin{chuy}{Lưu ý về chứng minh}{cy:chung-minh}
Chứng minh cho Hệ quả \ref{hq:vo-han-snt} sẽ được trình bày sau.
\end{chuy}

\end{lythuyetbox}

\section{Bài tập chương}

Đây là cách sử dụng môi trường bài tập và lời giải mới.

% --- HOW TO USE THE NEW SYSTEM ---
% The optional title is now in square brackets []. The label is in curly braces {}.
\begin{cauhoi}[Phân tích ra thừa số nguyên tố]{bt:phan-tich-1}
Hãy phân tích số 360 ra thừa số nguyên tố.
\end{cauhoi}

% The solution environment is now simpler than ever.
\begin{loigiai}
Ta có:
\[ 360 = 36 \times 10 = (6 \times 6) \times (2 \times 5) = (2 \times 3) \times (2 \times 3) \times (2 \times 5) \]
Sắp xếp lại các thừa số, ta được:
\[ 360 = 2^3 \times 3^2 \times 5^1 \]
\end{loigiai}


% This is an example of a question WITHOUT an optional title.
\begin{cauhoi}{bt:kiem-tra-snt}
Số 111 có phải là số nguyên tố không? Tại sao?
\end{cauhoi}

\begin{loigiai}
Không. Số 111 không phải là số nguyên tố.
Ta có tổng các chữ số của 111 là $1+1+1=3$. Vì tổng các chữ số chia hết cho 3, nên 111 chia hết cho 3.
Thật vậy, $111 = 3 \times 37$.
\end{loigiai}